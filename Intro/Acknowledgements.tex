I have this memory from Space Camp that stands out. We went into a dark room with an overhead projector. An old man started putting up slides about the Universe, starting from the Big Bang and going into the formation of the galaxies, stars, and Solar System. He identified himself as an astrophysicist and I remember thinking that would be a cool job to have. Now, 13 years later, I'm dumbfounded that I'm writing this memory down in my doctoral dissertation. I don't remember his name or know if he's even still alive, but I wanted to make him the first person I thanked for sparking my initial interest. I'd also like to thank the Boyum family, Mary Ann and Paul, for gifting me the scholarship in memoriam of their son, Brad, which would send me to that camp and ultimately on the road I would follow for the next half of my life.

Thank you to my dissertation committee members: Keren~Sharon, Chris~Miller, Jon~Miller, Gus~Evrard, and Jean-Paul~Kneib. I greatly appreciate the time they have taken to help and guide me over the last few years planning my dissertation, especially Keren, the person who has read the next 200-ish pages in more depth than any other person ever will. The chapters to come will show that I obviously have learned a great deal from Keren, who has advised and mentored me with my research during my time as her graduate student. I'd like to share here some things she taught me that you won't find in this dissertation.

Working with Keren, I second-handedly picked up new ways to be creative, clever and quick-witted. She always told me that my time as a researcher is invaluable and it's worth it to strive for a streamlined workflow. I think I've finally joined her at the ranks of black belt in ds9/XPA kung fu. The two of us were involved in several research groups outside of the University of Michigan throughout my graduate career. A valuable piece of advise Keren told me was that ``you can't pick your family, but you are free to choose your collaborators." As a Minnesotan, I find it difficult to say ``no" to others, with fear of experiencing that passive aggression we're known so fondly for. Keren told me to try to surround myself with people who I respect and who will give me respect in return. That being said, she introduced me to many awesome collaborators, a couple in particular I should take the time to thank now: Jane~Rigby and Mike~Gladders. Jane and Mike are enthusiastic and extremely adept and intelligent researchers who have greatly helped push my research forward, especially on the scientific-applications for lensing, as well as being quite fun to work with. And by happy coincidence, being in a research setting with two openly-gay and thriving astrophysicists provided me with much-needed role models as a junior scientist and as someone who was just beginning to step out of the closet. Finally, Keren taught me how to drive a stick shift, on a mountain in the Atacama Desert. It's tough to say whether or not that last one is more practical than the others. In all seriousness, Keren has been a brilliant, insightful, and entertaining mentor to work with. Through my ups and downs, she's always been dedicated to helping me succeed.

I couldn't have asked for a more supportive place to pursue my graduate studies than the University of Michigan. I will certainly miss all my friends with whom I have shared the thrills as well as the morbidity of graduate school. In somewhat chronological order, I'd like to thank Kamber~Schwarz, Vivienne~Baldassare, Rachael~Roettenbacher, Colin~Slater, Ilse~Cleeves, Aleksandra~Kuznetsova, Adi~Foord, Renee~Ludlam, and Juan~Remolina. I'd also like to add postdoc Rachel Paterno-Mahler to the that list as well. I'm extremely grateful to have found lifelong friends, fellow cat enthusiasts, drinking buddies, trivia geeks, social justice warriors, and SnapChat followers who continue to make my day-to-day life more lively.

I fell in love with Ann Arbor with the first step I took on the streets of downtown during my prospective visit. In the following years I began to fall in love with some of its more permanent residents. A thanks with a big hug and kiss to Roger Esterwood, who has become my best friend outside of astronomy, go-to happy hour attendee, political debater, pre-90s music educator, and brunch buddy in A2. And thanks to my other Aut Bar queers and allies: Rodger, Redcloud, Matt, Laura, Lee, Geoffrey, Steve, Nyci, Jeanette, Amy, Xi, Michael, Tom, Big Tim, Little Tim, and Billy, with whom I have been great companions for when the time comes to ``de-stress" with alcohol, loud music, and drag royalty.

Finally, I'd like to thank my parents for being so supportive of my path. My mother has always been ecstatic about having access to her own personal astrophysicist. My father took some convincing, as he initially wasn't thrilled with my choice of attending Carleton College, the most liberal and expensive school in the state of Minnesota, nor with a major like physics (Where's the money in that?). However, he eventually came around after seeing what I had accomplished academically, professionally, and on the golf course during my years at the college in their hometown of Northfield, which they lovingly refer to as the ``Harvard of the midwest" (though, folks in A2 would argue that's U. Michigan). He's now as equally as excited as my mother for the day when ``I get to wear those goofy robes" on graduation day. So~am~I.

I probably shouldn't thank my cat, Starbuck. She was too much of a distraction while writing this dissertation. When she wasn't constantly fighting with my computer over my lap she was just sitting there, competing with the screen for my attention. Lensing is beautiful, but she's so cute that there really is no contest. Thanks for being such a cuddly companion through all of it, buckaroo.
