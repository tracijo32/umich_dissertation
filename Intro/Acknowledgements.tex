When I attended Space Camp in high school, I realized there are about a million ways space can kill a human being, so it's safe to say I was turned off a bit from a career in space exploration. One afternoon, after several mock space shuttle missions, we took a break with a science lecture. We went into a dark room with an overhead projector. An old man started putting up slides about the universe, starting from the Big Bang and going into the formation of the galaxies, stars, and Solar System. He identified himself as an astrophysicist. Everyone else at the camp was falling asleep from exhaustion, but I was just waking up. From that moment on, I knew wanted to be an astrophysicist, and it's kind of crazy that I'm writing this now in the my doctoral dissertation in astrophysics. I don't remember his name or know if he's even still alive, but I wanted to make him the first person I thanked. I'd also like to thank the Boyum family, Mary Ann and Paul, for gifting me the scholarship in memoriam of their son, Brad, that would send me to that camp and ultimately on the road I would follow for the next decade of my life.

I'd like to thank my parents for being so supportive of my path. My mother has always been ecstatic about having access to her own personal astrophysicist. My father took some convincing, as he initially wasn't thrilled with my choice of attending Carleton College, the most liberal and expensive school in the state of Minnesota. However, after seeing what I had accomplished academically, professionally, and on the golf course during my years at the ``snobby elitist school" in their rural hometown of Northfield, he eventually came around. He's now as equally as excited as my mother for the day when ``I get to wear those goofy robes with the other doctor people" on graduation day.

As the next 250 pages will show, I obviously have learned a great deal from Keren Sharon, who has advised and mentored me with my research during my time as her graduate student. I'd rather share here the things I learned from her that you won't find in this dissertation. I learned to be clever and quick-witted; two skilled I never thought could be learned as an adult. Keren always told me that my time as a research is invaluable and should not be wasted on the mundane. From stitching together complex ray tracing code with a complicated affine-invariant MCMC ensemble sampler, to taking the helm of a supercomputer to create more lens models than any lens modeler has modeled before, to earning a ``black belt" in ds9 ``kung fu," I learned that there is almost always a better and more efficient way of doing things, I just have to ``think like Keren" and take the time to envision and develop it. The two of us were involved in several research groups outside of the University of Michigan throughout my graduate career. A valuable piece of advise Keren told me was that ``you can't pick your family, but you are free to choose your collaborators." As a Minnesotan, I find it difficult to say ``no" to others, especially those more senior than I. Keren told me to only surround myself with people with whom I respect and who will give me respect in return. That being said, she introduced me to many awesome collaborators, a couple in particular I should take the time to thank now: Jane Rigby and Mike Gladders. Jane and Mike are enthusiastic and extremely adept and intelligent research who have greatly helped push my research forward, especially on the scientific-application side of my research. And sort of by accident, being in a research setting with two openly gay and thriving astrophysicists provided me with much-needed role models as a junior scientist and as someone who was just beginning to step out of the closet. Keren one time told me once my first year I need to ``stop being shy;" easy for her to say as a very outgoing person compared to myself. However, I feel I've made significant progress towards overcoming social anxiety. It helps when someone like me has an advisor who helps meet you half way at conferences when it comes to meeting and networking with new people. Finally, Keren taught me how to drive a stick shift, on a mountain in the Atacama Desert. I guess when you are in the middle of nowhere, you need to learn to work with what you have. In all seriousness, Keren has been a brilliant, insightful, and entertaining mentor to work with. Through my ups and downs, she's always been dedicated to helping me succeed. Thanks, Keren!

I couldn't have asked for a more supportive place to pursue my graduate studies than the University of Michigan. I will certainly miss all my friends with whom I have shared the thrills as well as the morbidity of graduate school. In somewhat chronological order, I'd like to thank Kamber Schwarz, Vivienne Baldassare, Rachael Roettenbacher, Colin Slater, Ilse Cleeves, Aleksandra Kuznetsova, Adi Foord, Renee Ludlam, and Juan Remolina. I'm extremely grateful to have found lifelong friends, fellow cat enthusiasts, drinking buddies, trivia geeks, social justice warriors, and SnapChat followers who continue to make my day-to-day life more lively.

I fell in love with Ann Arbor with the first step I took on the streets of downtown during my prospective visit. However, it took a few years for me to begin to fall in love with some of its more permanent residents. A thanks with a big hug and kiss to Roger Esterwood, who has become my best friend outside of astronomy, go-to happy hour attendee, political debater, pre-90s music educator, and brunch buddy in A2 outside of astronomy. And thanks to my other Aut Bar queers and allies: Rodger, Redcloud, Matt, Laura, Lee, Geoffrey, Steve, Nyci, Jeanette, Amy, Hayes, Michael, and Billy, with whom I have been great companions for when the time comes to ``de-stress" with alcohol, loud music, and drag royalty.

My graduate school years were certainly memorable and have shaped me greatly into the person I am today perhaps more than any other time in my life, professionally as well as socially, personally, and politically. As I exit astronomy, I'll never forget the people who I met along the way. Having put forward this dissertation, which I have worked hard on and am quite proud of, I am excited to see what the future holds for me.