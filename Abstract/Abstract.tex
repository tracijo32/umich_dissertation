Massive galaxy clusters are capable of magnifying background sources and thus act as natural telescopes to the distant Universe. My dissertation focuses on modeling the mass distributions of these clusters in order to determine to what degree intrinsic properties of background sources such as luminosity, star formation rate, and size have been magnified. With accurate and precise lens models, we can compute the luminosity functions of the most distant galaxies $z>8$, pushing beyond the limits of \hst\ deep fields, which will help to understand the formation of galaxies during the epoch of re-ionization. We can also use cluster lensing to zoom into galaxies at $z\sim2$ to study their star formation morphologies on scales smaller than a kiloparsec, science that will not be feasible for field galaxies until {\it JWST} comes online. In addition to creating lens models, my dissertation also looks at the systematic errors associated with lens modeling techniques. As I will show, lens model accuracy can depend on the number of constraints as well as the availability of spectroscopic redshifts used in the modeling process. Understanding the systematic errors of lens models will be necessary in the next decade, which several wide-field surveys will reveal thousands of new strong lensing systems, for which higher resolution imaging and spectroscopic data may not be obtained for all systems.